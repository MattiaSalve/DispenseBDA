\documentclass{article}

\usepackage[italian]{babel}
\usepackage{amsfonts,multicol, amsmath, tikz, pgfplots, titlesec, fontspec, xcolor}
\pgfplotsset{compat=newest}
\usepackage[a4paper]{geometry}

\setmainfont{DINNextW1G}[
    Path = ./font/,
    Extension = .otf,
    UprightFont=*-Regular,
    BoldFont=*-Bold,
    ItalicFont=*-Italic,
    BoldItalicFont=*-BoldItalic,
]
\definecolor{ETHBlau}{RGB}{33, 92, 175}

\usepgfplotslibrary{fillbetween}

\pgfmathdeclarefunction{gauss}{2}{%
  \pgfmathparse{1/(#2*sqrt(2*pi))*exp(-((x-#1)^2)/(2*#2^2))}%
}

\begin{document}
    \begin{titlepage}
        \begin{center}
            \vspace{3cm}
            \Huge\textbf{Dispense di BDA}\\
            \vspace{1cm}
            \huge{A cura di Mattia Salvetti}\\
            \vspace{1cm}
            \large{Anno Accademico 2023/24}\\
            \vspace{5cm}
            \begin{tikzpicture}[]
                \begin{axis}[width = \textwidth, height = 0.5 * \textheight,
                    xmin = -11, xmax = 11, ymin = 0, ymax = 0.15, axis lines = middle,
                    xticklabels={},yticklabels={}]
                    \path [name path=xaxis]
                    (\pgfkeysvalueof{/pgfplots/xmin},0)
                    (\pgfkeysvalueof{/pgfplots/xmax},0);
                    \addplot [smooth, thick, domain = -10:10, name path = A] {gauss(0,3)};
                    \addplot [ETHBlau, opacity = 0.3]fill between [of = A and xaxis];
                \end{axis}
            \end{tikzpicture}
            \vspace{1cm}
            
        \end{center}
    \end{titlepage}
    \tableofcontents
    \newpage
    \begin{center}
        \Huge\textbf{Business Data Analytics I}
    \end{center}
    \section{Introduzione} 
    Lo scopo fondamentale di questo corso è quello di fornire lo studente con gli strumenti necessari a elaborare ed estrapolare informazioni utili da grandi moli di dati. Proprio la quantità di questi dati porta alla necessità di capire cose è importante e cosa no, e diventa fondamentale avere dei modelli per estrapolare informazioni automaticamente su basi statistiche. 
    \subsection{Dati Multivariati}
    Possiamo rappresentare i dati in forma matriciale, avendo $n$ unità statistiche di cui osserviamo $p$ variabili. A questo punto possiamo scrivere la matrice dei dati come 
    \begin{equation}
        \mathbb X = \begin{bmatrix} x_{11} & x_{12} & \ldots & x_{1p} \\ x_{21} & x_{22} & \ldots & x_{2p} \\ \vdots & \vdots & \ddots & \vdots \\ x_{n1} & x_{n2} & \ldots & x_{np} \end{bmatrix} \in \mathbb R^{n x p}
    \end{equation}
    dove $x_{ij}$ è la $j$-esima variabile della $i$-esima unità statistica. 
    Fatto ciò i dati possono essere rappresentati come una nuvola di punti appartenente a $\mathbb R^p$.
    Ricordiamo che al crescere di $p$ l'analisi diventa più complicata, esistono quindi dei metodi di riduzione dimensionali che permetto di passare da $p$ a $z$ variabili con $z < p$.
    
    \section{Analisi Univariata}
        Possiamo vedere la nostra matrice dei dati come una serie di vettori riga
        \begin{equation}
            \mathbb X =  [X_{1}, X_{2}, \dots, X_{n}]^T 
        \end{equation}
        Dove $X_1 = [x_{11}, x_{12}, \ldots, x_{1p}]$, $X_2 = [x_{21}, x_{22}, \ldots, x_{2p}]$ e così via.
        A questo punto, dai campioni si può fare statistica descrittiva o inferenziale. 
        Partiamo da una rapida discussione della statistica descrittiva.
        \subsection{Statistica Descrittiva}
        La statistica descrittiva è una statistica che ci è fornita una descrizione della distribuzione dei dati.
        Alcuni indici per la statistica descrittiva sono:
        \begin{itemize}
            \item Media campionaria: $\bar x_j = \frac{1}{n} \sum_{i=1}^n x_{ij}\ \forall j$
            \item Varianza campionaria: $S_{jj} = \frac{1}{n} \sum_{i=1}^n (x_{ij} - \bar x_j)^2$ NB: spesso si usa $n-1$ al denominatore per ragioni di non distorsione.
            \item Deviazione standard: $\sqrt{S_{jj}}$
        \end{itemize}

        Un altro aspetto che tornerà utile per i nostri scopi è lavorare con variabili confrontabili. Per fare ciò è fondamentale che esse siano standardizzate. 
        Standardizzare una variabile vuol dire renderla adimensionale e a media nulla, così facendo:
        \begin{equation}
        x^*_{.j}=\frac{x_{.j} - \bar x_j}{{\sqrt{S_{jj}}}}
        \end{equation}
        \subsection{Statistica Inferenziale}
        Da un campione casuale $x_1, \dots, x_n$ possiamo, ipotizzando che i dati siano indipendenti ed identicamente distribuiti, ovvero che ogni osservazione sia indipendente da quella precedente e che esse seguano tutte la stessa distribuzione statistica $X_1, \dots, X_n \sim \mathcal F(\theta)$, dove $\mathcal F (\theta)$ è la legge della variabile aleatoria, e $\theta$ è il parametro della famiglia di distribuzioni.
        Fatto ciò possiamo: 
        \begin{itemize}
            \item Stima puntuale: usare i parametri per associare un singolo valore al parametro
            \item Stima intervallare: usare i parametri per associare un intervallo di valori al parametro con un certo intervallo di confidenza, ovvero costruisco un intervallo di confidenza con un certo livello
            \item Test d'ipotesi: usa i parametri della distribuzione per capire se l'ipotesi $H_0$ è vera o falsa. 
        \end{itemize}
        \subsubsection{Stima Puntuale}



\end{document}
