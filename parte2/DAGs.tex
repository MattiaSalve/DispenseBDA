\selectlanguage{italian}
\chapter{DAGs}
Nel 2000 J. Pearl ha introdotto un chiaro e innovativo approccio grafico al tema della 
causalità: \textit{Directed Acyclic Graphs}.
L’obiettivo di un DAG è quello di disegnare un sistema causale per rappresentare 
esplicitamente tutte le cause dell’outcome d'interesse.
Il modello di causalità è semplificato in quanto: 
\begin{itemize}
    \item Assume un effetto omogeneo su tutte le osservazioni
    \item Utilizza solo outcome osservabili e non potenziali
    \item Si concentra sull'effetto medio incondizionato del trattamento
    \item Non specifica il tipo di relazione tra le variabili
\end{itemize}
Un po' di terminologia riguardante i grafici aciclici orientati: 
\begin{itemize}
    \item Orientato: tutte le relazioni puntano da una causa a un effetto
    \item Aciclico: partendo da un qualunque vertice non possiamo tornare allo stesso
    \item Percorso: una qualsiasi sequenza di vertici orientati in una qualsiasi direzione
    \item Percorso diretto (o causale): un percorso in cui tutti i nodi puntano in una sola 
        direzione
\end{itemize}
Vengono inoltre detti: 
\begin{itemize}
    \item Genitore: una causa diretta di una variabile
    \item Figlio: l'effetto diretto di una certa variabile
    \item Antenato: una causa diretta o indiretta di una certa variabile
    \item Discendente: un effetto diretto o indiretto di una certa variabile
\end{itemize}
\subsection{DAG - Modello di Markov}
\begin{marginfigure}
    \includegraphics[width = \textwidth]{dag}
\end{marginfigure}
Un DAG è un modello in cui le variabili dipendono solo dai propri genitori e sono 
indipendenti dalle altre variabili. 
La funzione di probabilità di massa è 
\begin{equation}
    f(x) = \Pi_{i = 1}^N f(x_i|x_{iPA})
\end{equation}
\\Se la relazione tra A e B è mediata da C, C è detto mediator\\
Se A e B concausano C, C è un collider\\
Se A e B sono causati da C, C è un cofounder\\
Attenzione perché il cofounder è esattamente ciò che crea problemi nell'analisi: infatti è 
facile confondere la correlazione tra A e B per causalità, mentre la causa della correlazione
è C.
Un percorso causale può essere immediato $A \to B$ o mediato, come per esempio $A \to B \to C$.
L'effetto totale di A su B è la combinazione di tutti i percorsi immediati e mediati da A a B.

L'utilizzo dei DAG permette di offrire una rappresentazione delle relazioni causali,
chiarire le domande di ricerca ed evidenziare i concetti rilevanti, rendere esplicite le 
assunzioni dei nostri modelli, indentificare appropriatamente le variabili da inserire nell’analisi
e ottenere risultati più affidabili, riducendo possibili bias.
Per costruire un DAG: 
\begin{itemize}
    \item Articolare la domanda di ricerca identificando la causa e l’effetto a cui si è 
        interessati («qual è l’effetto di A su B?»)
    \item Identificare altre variabili rilevanti per la relazione, come collider e mediator
    \item Identificare variabili cofounder
    \item Identificare eventuali variabili non misurate o non misurabili
    \item Identificare possibili processi di selezione nello studio
\end{itemize}
Negli anni sono stati creati strumenti a supporto della costruzione di un DAG come 
\url{http://www.dagitty.net/}
\vspace{0.5cm}

\large Paradosso di Simpson
\normalsize
\\Situazione statistica nella quale un trend o una relazione che è osservata tra diversi 
sottogruppi sparisce quando i gruppi sono combinati. In altre parole, dividendo i dati in 
gruppi, le conclusioni sono diverse rispetto a quelle derivanti da un’analisi aggregata.
Una back-door path (BDP) è un percorso che è diretto verso D da una parte e termina verso Y
dall'altra, cioè si ha una connessione tra D ed Y che non segue il percorso delle frecce.
Lasciare un BDP aperto introduce un bias e non permette di indentificare l'effetto causale
per via della presenza del cofounder. Per identificare la causalità, bisogna quindi chiudere
tutti i BDP aperti. 
\begin{marginfigure}
    \includegraphics[width = \textwidth]{BDPchiuso}
    \caption{Sulla sinistra, un BDP aperto. Sulla destra, controllando C, eliminiamo l'impatto 
        di C su D ed A. La relazione fra D e Y è ora determinata solo da un percorso diretto.}
\end{marginfigure}
In questo esempio avremmo potuto controllare sia per C che per A, identificando correttamente
la causalità tra D ed Y, ma allora perché non controllare per tutte le variabili sempre, senza
perdere tempo capendo le relazioni tra le varaibili?
Perché controllare variabili senza criterio può invalidare l'analisi qualora: 
\begin{itemize}
\begin{marginfigure}
    \includegraphics[width = \textwidth]{overcontrolling}
    \caption{Se controllassimo per C, perderemmo l'intero (caso 1) o parte (caso 2) impatto
    di D su Y}
\end{marginfigure}
    \item La variabile controllata sia un mediator, cioè un discendente di D in un percorso 
        diretto verso Y
\begin{marginfigure}
    \includegraphics[width = \textwidth]{overcontrollingcollider}
    \caption{Controllare un collider può portare a creare un'associazione fra le variabili che 
    lo concausano, anche quando queste sono indipendenti tra loro. Questo può portare all'apertura
di un percorso controllando per un collider.}
\end{marginfigure}
    \item La variabile controllata sia un collider in una back-door path da D a Y
\end{itemize}
Per fare un esempio si supponga che il merito creditizio (A) e la validità di un progetto (B)
influenzino la ricezione di un prestito (P = 1 se il prestito è concesso, 0 altrimenti). 
Ipotizzando che A e B siano indipendenti e controllando per P osserveremmo una falsa correlazione 
tra A e B. In particolare, se selezioniamo solo le imprese con P = 1, se esse hanno un progetto
poco convincente dovranno avere un merito creditizio molto alto, creando una relazione di correlazione
inversa tra le due variabili incorrelate.
\subsubsection{Criterio del back-door}
Condizioni sufficienti per l'identificazione dei nessi causali nei DAG:
\begin{itemize}
    \item Identificare tutti i back-door path da D a Y
    \item Identificare i mediator
    \item Bloccare solo i BDP aperti, controllandone le variabili cofounder
    \item Stare attento a non controllare per mediator e collider
\end{itemize}
\subsubsection{Criterio del front-door}
Potrebbe esistere una variabile U non osservabile (e quindi non controllabile), allora il 
BDP $D \leftarrow U \to Y$ non può essere chiuso.
\begin{marginfigure}
    \includegraphics[width = \textwidth]{frontdoor}
\end{marginfigure}
Nell'esempio di destra, posso semplicemente stimare l'effetto di D su M ed N, e poi di queste 
su Y. Combinando l'effetto delle due avremo l'effetto di D su Y. Per poter applicare il 
criterio del front-door, le variabili M ed N devono essere 
\begin{itemize}
    \item Esaustive: la combinazione di M ed N catturano tutto l'effetto di D su Y
    \item Isolate: tutti i back-door path da M a N sono bloccati un volta controllata D (
        cioè non esistono cofounder per M e N)
\end{itemize}
