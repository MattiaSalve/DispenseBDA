\selectlanguage{italian}
\chapter{Introduzione parte II}
Questa parte di corso si concentra sul condurre inferenza su i dati disponibili.
In particolare, la statistica classica ci può aiutare a trovare correlazione tra i dati,
ci avverte che correlazione non è causalità\sidenote{Per esempio, se i lavoratori $L$ di 
un'azienda sono proporzionali al fatturato $F$ tramite una costante di proporzionalità $k$
, allora $L = kF$, $L - kF = 0$ e $F = L/k$ sono tutte corrette, non abbiamo informazioni
su quale sia il driver, se il fatturato o il numero di lavoratori.}, ma non ci dice quale 
sia la causa di un fenomeno e come si identifica.
Le risposte a queste domande non risiedono nei meri dati.

