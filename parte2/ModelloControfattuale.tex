\selectlanguage{italian}
\chapter{Modello del Controfattuale}

Dati:\\ 
$Y_{0i}$ = Stato di salute della persona $i$ senza copertura assicurativa \\
$Y_{1i}$ = Stato di salute della persona $i$ con copertura assicurativa \\
L'effetto causale della copertura assicurativa è $Y_{1i} - Y_{0i}$. Se potessi osservare 
la stessa persona potrei così stimare l'impattto della copertura assicurativa. Questo è impossibile
nella realtà, quindi ci limitiamo a confrontare persone diverse con e senza polizza.
Supponiamo quindi di osservare due persone, $i$ che sottoscrive la polizza (trattata) e 
$j$ no (controllo). Nella realtà conosciamo solo $Y_{1i}$ e $Y_{0j}$, e non le altre. Un 
modo per confrontare la validità della polizza può essere $Y_{1i} - Y_{0j}$, che può essere 
riscritto come: $Y_{1i} - Y_{0j} = Y_{1i} - Y_{0i} + Y_{0i} - Y_{0j}$, ovvero l'effetto causale
del trattamento su $i$, più la differenza tra $i$ e $j$ se entrambi scegliessero di non 
sottoscrivere la copertura assicurativa. Questo secondo termine descrive la fragilità di $i$
rispetto a $j$, ed è ciò che, nell’obiettivo d'individuare la relazione causale tra trattamento 
e outcome, viene definito come \textit{selection bias} 
Il confronto univariato quindi non permette di valutare in modo corretto la relazoine causale
tra trattamento e outcome a causa del selection bias, ovvero della differenza tra trattate
e controllo prima del trattamento.

Estendendo le nostre considerazioni a un campione di trattate e controllo composti da più
di un'osservazione:
\begin{equation}
    \mathbb{E}(Y|D=1) - \mathbb{E}(Y|D=0) = \mathrm{Effetto\ causale}(k) + \mathrm{selection\ bias}
\end{equation}
Se l'unica ragione di selection bias fossero differenze che possono essere misurate, allora
non è un problema utilizzare il modello controfattuale: infatti basta prendere campioni caratterizzati
da trattate simili.
Il trattamento può essere rappresentato con una variabile dicotomica: $D_i=1$ se $i$ è trattato,
$D_i=0$ altrimenti. Quindi il gruppo dei $D_i=0$ è definito gruppo di controllo (C), mentre il 
gruppo dei $D_i = 1$ è il gruppo di trattamento (T).

Definiamo l'\textit{outcome potenziale}:
\\$Y_{1i}=$ outcome potenziale se l'unità $i$ è trattata ($D_i = 1$)
\\$Y_{0i}=$ outcome potenziale se l'unità $i$ non è trattata ($D_i = 0$
\\$Y_i = Y_{0i} + (Y_{1i} - Y_{0i})D_i$

L'impatto causale è $Y_{1i} - Y_{0i}$, ma non possiamo osservare entrambi per uno stesso campione,
si necessita quindi di un controfattuale (ovvero un'approssimazione di $Y_0$) per valutare 
l'efficacia del trattamento: \textit{problema fondamentale dell'inferenza causale}.
\\Per ottenere la stima analizziamo molteplici unità per ricavare l'effetto causale aggregato,
se l'effetto del trattamento è omogeneo, corrisponde all'effetto individuale (ovvero si può
assumere k costante).
Perché questo approccio sia valido occorre che non ci siano effetti di spillover tra gruppo 
di controllo e di trattamento. In caso di spillover, l’outcome del gruppo di controllo è 
influenzato da quello del gruppo trattamento, comportando un bias nel risultato.
SUTVA (Stable Unit treatment value assumption): l’outcome potenziale di qualsiasi unità $i$ 
non varia con il trattamento assegnato ad altri, e per ogni unità non ci sono differenti 
versioni dello stesso trattamento che porta a differenti outcome potenziali.
Un esempio di spillover può essere un gruppo di dipendenti che segue un corso di formazione
e poi condivide le informazioni con altri dipendenti che non lo hanno seguito.
\subsection{Eterogeneità del trattamento}
ATE (Average Treatment Effect):
\begin{equation}
   \mathbb E = \mathbb E(Y_1 - Y_0) = \mathbb E(Y_1) - \mathbb E(Y_0)
\end{equation}
ATT (Average Treatment Effect on Treated):
\begin{equation}
   \mathbb E(\delta|D=1) = \mathbb E(Y_1-Y_0|D=1) - \mathbb E(Y_0|D=1)
\end{equation}
ATU/ATNT(Average Treatment Effect on Untreated):
\begin{equation}
   \mathbb E(\delta|D=0) = \mathbb E(Y_1-Y_0|D=0) = \mathbb E(Y_1|D=0) - \mathbb E (Y_0|D=0)
\end{equation}

L'ATE è una combinazione lineare di ATT e ATU. Indicando con $\mu$ la percentuale di popolazione
trattata: $\mathrm{ATE} = \mu \mathrm{ATT} + (1 - \mu) \mathrm{ATU}$, che può essere scritto
anche come \begin{equation}
   \mathbb E(\delta) = Pr(D=1)\mathbb E (\delta|D=1) +Pr(D=0)\mathbb E(\delta | D=0)
\end{equation}
Possiamo stimare ATT e ATU utilizzando uno stimatore ingenuo (\textit{naive estimator})
\begin{equation}
   NE = \mathbb E (Y_1|D=1) - \mathbb E(Y_0|D=0)
\end{equation}
\begin{equation}
   NE = \mathbb E (Y_1|D=1) - \mathbb E(Y_0|D=1) + \mathbb E(Y_0|D=1) - \mathbb E (Y_0|D=0)
\end{equation}
Dove i primi due termini sono l'ATT e gli ultimi due il Bias, che indica quanto sbaglieremmo
se approssimassimo ciò che non osserviamo $\mathbb E (Y_0|D=1)$ con ciò che osserviamo 
$\mathbb E (Y_0|D=0)$
\begin{equation}
   NE = \mathbb E (Y_1|D=0) - \mathbb E(Y_0|D=0) + \mathbb E(Y_1|D=1) - \mathbb E (Y_1|D=0) = ATU + BIAS
\end{equation}
Conseguentemente, anche l'ATE stimato usando l'NE non sarà privo di bias, in quanto 
\begin{equation}
    NE = \pi(ATT + BIAS_{NEATT}) + (1-\pi)(ATU + BIAS_{NEATU})
\end{equation}
